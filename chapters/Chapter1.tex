% !TEX root = ../main.tex
\chapter{绪论}

\section{研究意义与背景}
深度估计作为三维计算机视觉领域的核心基础问题,旨在从二维图像中重建场景的几何特征,在过去十余年中经历了从手工特征建模到深度学习驱动的范式变迁\cite{ref47}。
随着传感器技术与计算能力的提升,深度估计已成为智能无人系统实现环境感知与定位导航的关键技术:
在自动驾驶中,它为障碍物检测与路径规划提供必要的距离信息;在无人机技术中,它是实现自主避障与三维测绘的基础;在具身智能领域,深度估计则赋予了智能体感知空间结构并进行物理交互的能力。

从技术演进的维度看,单目深度估计的研究历程主要经历了三个阶段。初期阶段主要依赖手工设计的几何特征与先验假设 \cite{ref46},
通过概率图模型整合图像的底层信息,但在复杂场景下的建模鲁棒性较差。
中期阶段随着卷积神经网络(CNN)的兴起,研究重心转向端到端的监督学习 \cite{ref8},通过多尺度网络架构显著提升了像素级的预测精度。
现阶段则迈入了以大模型和多任务迁移为特征的新时期,Vision Transformer (ViT) 等架构的应用 \cite{...} 极大增强了模型对全局几何上下文的理解。

然而,现有的深度估计方法在非约束场景下的表现仍面临严峻挑战。由于单目深度估计本质上是一个不适定问题,存在固有的比例模糊性\cite{...},
神经网络往往倾向于通过学习训练集中的统计偏见(如物体位置与深度的相关性)来“走捷径”,而非真正理解场景的物理几何。这种策略导致模型过度拟合了特定数据集(如 KITTI 或 NYU Depth V2)的成像特性,使其对相机内参及拍摄视角具有极强的依赖性。一旦应用于光照剧变、极端天气或异质场景,由于领域鸿沟的存在,模型的预测精度往往会出现断崖式下降,限制了其在跨平台部署时的零样本迁移能力。

针对上述挑战,学术界开始探索一种新的研究范式:利用强泛化性的相对深度信息辅助绝对深度的估计。 
相对深度虽然不具备物理单位,但其能通过海量异质数据的预训练,捕捉到稳健的几何拓扑关系与遮挡先验,展现出极佳的场景鲁棒性。

本文认为,融合相对深度的几何先验优势与绝对深度的尺度特性,是实现跨场景稳定感知的关键路径。 
通过设计一种能够解耦几何结构与物理尺度的预测框架,利用大模型提取的全局几何一致性来约束局部尺度恢复,可显著降低模型对特定相机内参的耦合。
这种方法旨在打破单目深度估计在未知场景下的精度瓶颈,为无人系统在复杂、全天候环境下的高精度感知提供新的理论支撑与技术方案。
\section{国内外研究现状}
\subsection{利用相对深度估计}
